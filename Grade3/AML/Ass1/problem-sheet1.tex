% 请确保文件编码为utf-8,使用XeLaTex进行编译,或者通过overleaf进行编译

\documentclass[answers]{exam}  % 使用此行带有作答模块
% \documentclass{exam} % 使用此行只显示题目

\usepackage{xeCJK}
\usepackage{zhnumber}
\usepackage{graphicx}
\usepackage{hyperref}
\usepackage{amsmath}
\usepackage{booktabs}
\usepackage{enumerate}
\usepackage{bbm}
\usepackage{color}

\pagestyle{headandfoot}
\firstpageheadrule
\firstpageheader{南京大学}{高级机器学习}{习题集一}
\runningheader{南京大学}
{高级机器学习}
{习题集一}
\runningheadrule
\firstpagefooter{}{第\thepage\ 页(共\numpages 页)}{}
\runningfooter{}{第\thepage\ 页(共\numpages 页)}{}

% no box for solutions
% \unframedsolutions

\setlength\linefillheight{.5in}

% \renewcommand{\solutiontitle}{\noindent\textbf{答:}}
\renewcommand{\solutiontitle}{\noindent\textbf{解:}\par\noindent}

\renewcommand{\thequestion}{\zhnum{question}}
\renewcommand{\questionlabel}{\thequestion .}
\renewcommand{\thepartno}{\arabic{partno}}
\renewcommand{\partlabel}{\thepartno .}


\begin{document}
\normalsize

\begin{questions}
\question [30] \textbf{机器学习导论复习题(前八章)}

高级机器学习的课程学习建立在机器学习导论课程的基础之上,从事机器学习行业相关科研工作需要较为扎实的机器学习背景知识。下面的题目则对机器学习基础知识进行复习。

\begin{parts}
	\part [10] 在现实中的分类任务通常会遇到类别不平衡问题,即分类任务中不同类别的训练样例数目类别差别很大,很可能导致模型无法学习。(a) 请介绍类别不平衡学习常用的策略。 (b) 假定当前数据集中每个样本点包含了$\left(\boldsymbol{x}_{i}, y_{i}, p_{i}\right)$,其中$\boldsymbol{x}_i, y_i, p_i$分别表示第$i$个样本的特征向量,类别标签,和样本的重要程度,$0 \leq p_{i} \leq 1$。对于SVM,任意误分样本点$\boldsymbol{x}_i$的惩罚用$p_i$代替,请在西瓜书p130页公式(6.35)的基础上修改出新的优化问题,并给出对偶问题的推导。
	
	\part [20] 通常情况下,模型会假设训练样本所有属性变量的值都已被观测到,但现实中往往会存在属性变量不可观测,例如西瓜根蒂脱落了,就无法观测到该属性值,此时问题就变成了有``未观测"变量的情况下,对模型参数进行估计。EM(Expectation-Maximization)算法为常用的估计参数隐变量的方法。(a) 假设有3枚硬币,分别记作A,B,C。这些硬币正面出现的概率分别是$a,b,c$. 进行如下投掷实验:先投掷硬币A,根据其结果选出硬币B或者硬币C,正面选硬币B,反面选硬币C;然后投掷选出的硬币,投掷硬币的结果,出现正面记作1,出现反面记作0;独立地重复n次实验。假设只能观测到投掷硬币的结果,不能观测投掷硬币的过程。问如何估计三硬币正面出现的概率。请基于EM算法思想详细地写出E步和M步的推导步骤。(b)经典的聚类算法K-means就是EM算法的一种特殊形式,K-means也被称为hard EM。请使用EM算法的思想解释K-means,并对比K-means和EM算法的不同之处。
\end{parts}

\begin{solution}
	\begin{parts}
		\part 
		(a)我们可以采用“再缩放”的思想,具体做法可以有如下几种:\\
		一、欠采样:减少训练集中的多数类样本使得各个类别样本数目相近,再进行学习,如EasyEnsemble和BalanceCascade算法。\\
		二、过采样:即增加一些少数类样本使得各个类别样本数目接近,然后再进行学习,如SMOTE算法等。\\
		三、阈值移动:在数据不平衡时,默认的阈值会导致模型输出倾向于类别数据多的类别,阈值移动通过改变决策阈值来偏重少数类。
		比如线性分类器$y=w^Tx+b$就是用计算出的$y$与阈值(默认0.5)比较,大于则分为正例,反之则分为反例。为了避免类别不平衡的影响,我们对决策规则做如下修正:
		$$\frac{y}{1-y}>1,\mbox{预测为正}\Rightarrow\frac{y}{1-y}*\frac{m^-}{m^+}>1,\mbox{预测为正}$$
		其中$m^+,m^-$分别是样本中正例,反例的数目\\
		此外,对于上式中的$m^-/m^+$,若用$cost^+/cost^-$(正例误分为反例的代价/反例误分为正例的代价)代替,就是代价敏感学习。\\

		(b)优化问题为:
		\begin{align*}
			&\min\limits_{\boldsymbol{w},b,p_i}\ \frac{1}{2}||\boldsymbol{w}||^2+C\sum\limits_{i=1}^mp_i\zeta_i\\
			&\ s.t.\ \ \ \ y_i(\boldsymbol{w}^T\boldsymbol{x_i}+b)\geq1-\zeta_i,\ i=1,2,...,m\\
			&\ \ \ \ \ \ \ \ \ \zeta_i\geq0,\ i=1,2,...,m
		\end{align*}

		对偶问题的推导:\\
		先写出拉格朗日函数$$L(\boldsymbol{w},b,\boldsymbol{\alpha},\boldsymbol{\zeta},\boldsymbol{\mu})=\frac{1}{2}||\boldsymbol{w}||^2+C\sum\limits_{i=1}^mp_i\zeta_i+\sum\limits_{i=1}^m\alpha_i(1-\zeta_i-y_i(\boldsymbol{w}^T\boldsymbol{x}_i+b))-\sum\limits_{i=1}^m\mu_i\zeta_i$$
		其中$\alpha_i\geq0,\ \mu_i\geq0$是拉格朗日乘子\\
		接下来分别对$\boldsymbol{w},b,\zeta_i$求偏导并令其$=0$可得
$$
			 \begin{cases}
				 \boldsymbol{w}=\sum\limits_{i=1}^m\alpha_iy_i\boldsymbol{x}_i \\
				 0=\sum\limits_{i=1}^m\alpha_iy_i \\ 
				 Cp_i=\alpha_i+\mu_i
			 \end{cases}
$$将这三个式子代入拉格朗日函数即可得到对偶问题:


\begin{align*}
	&\max\limits_{\boldsymbol{\alpha}}\ \sum\limits_{i=1}^m\alpha_i-\frac{1}{2}\sum\limits_{i=1}^m\sum\limits_{j=1}^m\alpha_i\alpha_jy_iy_j\boldsymbol{x}^T_i\boldsymbol{x}_j\\
	&s.t.\ \ \ \sum\limits_{i=1}^m\alpha_iy_i=0\\
	&\ \ \ \ \ \ \ 0\leq\alpha_i\leq Cp_i,\ i=1,2,...,m
\end{align*}

		\part
		(a)记参数$\theta=(a,b,c)$,硬币A的投掷结果无法观测,为隐变量$Z=(Z_1,Z_2,...,Z_n)^T$,第二次投掷的硬币结果为可观测变量$Y=(Y_1,...,Y_n)^T$\\
		在一次投掷中,得到结果$y=0,1$的概率$P(y|\theta)=ab^y(1-b)^{1-y}+(1-a)c^y(1-c)^{1-y}$\\
		我们可以写出似然函数$$P(Y|\theta)=\prod\limits_{j=1}^n[ab^{y_j}(1-b)^{1-y_j}+(1-a)c^{y_j}(1-c)^{1-y_j}]$$
		需要求解参数的最大似然估计$$\hat{\theta}=arg \max\limits_{\theta}logP(Y|\theta)$$
		首先对参数进行初始化(随机初始化即可),得到$\theta^{(0)}=(a^{(0)},b^{(0)},c^{(0)})$,接下来按照如下步骤迭代,
		直到收敛或者达到迭代次数为止,记第i次迭代时的参数为$\theta^{(i)}=(a^{(i)},b^{(i)},c^{(i)})$,第i+1
		次迭代如下:\\
		\textbf{E步}:计算在参数$\theta^{(i)}$下观测到的数据$y_j$来自硬币B的概率
		$$\mu_j^{(i+1)}=\frac{a^{(i)}(b^{(i)})^{y_j}(1-b^{(i)})^{1-y_j}}{a^{(i)}(b^{(i)})^{y_j}(1-b^{(i)})^{1-y_j}+(1-a^{(i)})(c^{(i)})^{y_j}(1-c^{(i)})^{1-y_j}}$$
		\textbf{M步}:对参数值进行更新\\
		\begin{align*}
			&a^{(i+1)}=\frac{1}{n}\sum\limits_{j=1}^n\mu_j^{i+1}\\
			&b^{(i+1)}=\frac{\sum\limits_{j=1}^n\mu_j^{(i+1)}y_j}{\sum\limits_{j=1}^n\mu_j^{(i+1)}}\\
			&c^{(i+1)}=\frac{\sum\limits_{j=1}^n(1-\mu_j^{(i+1)}y_j)}{\sum\limits_{j=1}^n(1-\mu_j^{(i+1)})}
		\end{align*}
		(b)K-means在一次迭代中的两步,可以分别看作是EM算法的E步和M步。\\
		其中,把各样本点依据到簇中心的距离分到最近的簇对应于E步\\
		更新各簇的均值向量作为新的簇中心对应于M步\\
		区别:K-Means算法中每次对参数的更新是硬猜测,而EM中每次对参数的更新是软猜测
	\end{parts}
\end{solution}


\question [25] \textbf{主成分分析}

主成分分析(Principal Component Analysis, PCA)是一种经典的无监督降维技术,可以有效减少数据维度,避免维度灾难。实际上,涉及PCA的算法有非常多,下面的题目将逐步引入更多关于PCA的内容。

\begin{parts}
	\part [5+5] 关于PCA,教材中给出了最近重构性和最大可分性两种推导方法,但是该方法将多个主成分在一起推导。实际上,有另外一种Step-by-step的推导方法更为具体。假设数据矩阵$X\in \mathcal{R}^{n\times d}$包含$n$个$d$维度的样本,每个样本记作$x_i \in \mathcal{R}^d$。下面基于Step-by-step的最大可分性进行推导。最大可分性的假设偏好是:样本在低维空间尽可能分散。(a) 假设选取第一个主成分为$w \in \mathcal{R}^d$,需要满足$\lVert w \rVert_2^2 = 1$,那么样本投影到该主成分的投影点为$w^Tx_i$,然后我们需要最大化投影点之间的方差,试写出具体的优化目标,并分析其与瑞利商(Rayleigh quoient)的关系。可假设数据已经中心化。(b) 在选取第一个主成分$w$之后,需要求解第二个主成分$v$,要满足和第一个主成分向量正交,即$v^Tw=0$,此时可以考虑将样本$x_i$分解为两个成分:沿着$w$的向量和垂直于$w$的向量。最后只需要对于垂直的部分选取第二个主成分即可。试给出具体的分解方法以及后续选取第二个主成分的推导过程。

	\part [5+5] 假设PCA得到的映射矩阵(主成分组成的矩阵)为$W\in \mathcal{R}^{d \times d^{\prime}}$,那么对数据矩阵$X\in \mathcal{R}^{n\times d}$降维的过程是:$XW \in \mathcal{R}^{n \times d^{\prime}}$。该过程可以看作是神经网络中不带有偏置(bias)的一层全连接映射。那么:(a) 基于最近重构性的PCA推导方法和AutoEncoder有什么关系?试分析二者的区别和联系(可以从公式、优化、实验效果等角度进行分析)。(b) 一般地,在深度神经网络中,对于全连接层会加入正则化项,例如二范数正则化$\lVert W \rVert_2^2$,在PCA中是否可以同样地对$W$施加正则化项呢?试给出具体的优化目标以及大概如何求解。(可参考Sparse PCA相关内容,只需说出求解优化问题的方法,无需给出具体求解算法和过程)。

	\part [5] (任选一题) 上题谈到了PCA和深度神经网络,我们知道深度神经网络一般基于梯度自动回传来进行反向传播,其自动梯度计算过程在PyTorch、Tensorflow等工具包中已经被实现。试问:(a) 请调研sklearn中实现的SVD的方法,试比较其提供的FullSVD、TruncatedSVD、RandomizedSVD等SVD的区别,如果有实验效果对比图(性能、运行效率)则更佳。(b) 试问在PyTorch中是否可以对SVD进行自动计算梯度,如有,请简单介绍其原理。
\end{parts}

\begin{solution}
	\begin{parts}
		\part 
		(a)\\
		我们的目标是最大化方差$$Var(w^Tx_i)=\frac{1}{m}\sum\limits_{i=1}^m(w^Tx_i-\frac{1}{m}(\sum\limits_{i=1}^mw^Tx_i))^2$$
		考虑到样本已经中心化,即均值为零,忽略系数后我们要最大化$\sum\limits_{i=1}^m(w^Tx_i)^2$\\
		此时优化问题为:\\
		\begin{align*}
			&\max\limits_{w}\ \sum\limits_{i=1}^m(w^Tx_i)^2=w^TX^TXw\\
			&s.t.\ \ \ w^Tw=1
		\end{align*}
		再做整理可得$$w=\arg\max\frac{w^TX^TXw}{w^Tw}$$
		等式右边即为瑞利商,最大值即为$X^TX$的最大的特征值,此时$w$为对应的单位特征向量\\

		(b)\\
		对于向量$x$,其与$w$平行的分量为$w^Tx_i*w$,故对所有$x_i$减去与$w$平行的部分,得到$x_i'=x_i-w^Tx_i*w$\\
		(分解方法为$x_i=w^Tx_i*w(\mbox{平行于w})+(x_i-w^Tx_i*w)(\mbox{垂直于w})$)\\
		此时对应的数据矩阵$X'=X-Xww^T$,该矩阵中保留了全部垂直于$w$的向量,故对该部分数据选取第二主成分。\\
		我们有$$v=\arg\max\frac{v^T(X')^TX'v}{v^Tv}$$可以求得$v$是$(X')^TX'$的最大特征值对应的单位特征向量,即$X^TX$的第二大特征值对应的单位特征向量。\\

		\part (a)\\
		(1)、这二者都可以用来降低数据维数,但是PCA无法用于学习非线性特征,AutoEncoder可以。\\
		(2)、PCA在降维后,必然存在信息损失,但是
		Autoencoder降维的方法是对数据进行数据编码再进行解码,之后最小化原数据与解码数据之间的误差
		从而学习到变换矩阵,这种方法虽然损失很小,但要付出更多的计算量。\\
		(3)、从公式上看,基于最近重构性的PCA要最小化函数$$\sum\limits_{i=1}^m||\sum\limits_{j=1}^{d'}z_{ij}\boldsymbol{w}_j-\boldsymbol{x}_j||_2^2$$
		其中,$d'$为降维后的空间的维数,$\boldsymbol{w_j}$是低维空间的第j个标准正交基向量,$z_{ij}$为原样本点投影到新坐标系的相应系数。\\
		而Autoencoder则是最小化$$\sum\limits_{i=1}^m||x_i'-x_i||^p_p$$
		其中$x_i$是原数据,$x_i'$是编码后再解码的数据,求解时通常基于反向传播算法。\\
		(b)优化问题如下:\\
		\begin{align*}
			&\arg\min\limits_{A,B}\ \sum\limits_{i=1}^n||x_i-AB^Tx_i||^2_2+\lambda_1\sum\limits_{j=1}^k||\beta_j||^2_2+\lambda_2\sum\limits_{j=1}^k||\beta_j||_1\\
			&s.t.\ \ \ \ \ \ \ A^TA=I^{k*k}
		\end{align*}\\
		其中$A=[\alpha_1,\alpha_2,...,\alpha_k]$为前k个非稀疏主成分,$B=[\beta_1,...,\beta_k]$为前k个稀疏主成分。\\
		求解方法大致为:\\
		1、固定A,利用$L_1$正则化求解线性回归问题从而得到B\\
		2、固定B,利用$X^TXB=U\Sigma V^T,A^*=UV^T$求解最优解$A^*$。

		\part (a)区别如下:\\
		FULLSVD是直接计算$m*n$矩阵$A$的奇异值分解($A=U\Sigma V^T$),返回$U,s,V^T$的结果($\Sigma$作为列向量$s$返回),
		并且对$s$中的奇异值从大到小排序,fullsdiag()可以将列向量s转化为奇异值分解中的矩阵$\Sigma$\\
		
		TruncatedSVD是实现SVD的一种变体,通过截断奇异值分解进行降维,它只计算k(由用户指定)个最大奇异值,即$X\approx X_k=U_k\Sigma_kV_k^T$,
		当应用于单词-文本矩阵时,这种变换即为潜在语义分析(LSA)。并且,这种变换适用于任何特征矩阵。\\

		RandomizedSVD用于计算截断的随机SVD,通过随机化来找到(通常表现非常好的)近似截断奇异值分解来加速计算,尤其是在只保留少量特征的大型矩阵上有着优异的表现。\\
		下表是对于一个随机生成的$1500*1500$矩阵,三种方法作奇异值分解各自消耗的时间(截断奇异值分解取k=300,表现最佳的数据用红色表示):\\

			\begin{center}
			  \begin{tabular}{lccc}
				\hline
				\textbf{实验次数} & \textbf{FullSVD} & \textbf{TruncatedSVD} & \textbf{RandomizedSVD}\\
				\hline
				1 & 2811.2144ms & 1091.3017ms & \textcolor{red}{838.4232ms} \\
				\hline
				2 & 2269.7217ms & 888.2902ms & \textcolor{red}{633.3055ms}\\
				\hline
				3 & 2992.6169ms & 1138.2227ms & \textcolor{red}{632.8134ms}\\
				\hline
				4 & 3148.6489ms & 1132.6401ms & \textcolor{red}{666.2230ms}\\
				\hline
				5 & 3089.3362ms & 1359.3519ms & \textcolor{red}{812.2656ms}\\
				\hline
			  \end{tabular}
			\end{center}
		实验体现出来的运行性能和三种SVD的描述相对应的情况还是很吻合的,FullSVD最慢,截断奇异值分解次之,随机截断奇异值分解最快。
	\end{parts}
\end{solution}

\question [15] \textbf{降维与度量学习}

降维与度量学习包含多种算法,例如PCA、NCA、LLE、MDS等等。接下来的几个题目会拓展大家对这些算法的认知范围。最后两道任选一题即可。
\begin{parts}
	\part [5] 近邻成分分析(Neighbourhood Component Anslysis, NCA)是基于KNN分类器的有监督降维算法。其优化目标主要是:$f=\sum_{i=1}^n p_i = \sum_{i=1}^n \sum_{j \in C_i} p_{ij}$,其中$C_i=\{j|y_j=y_i\}$表示与第i个样本类别一样的下标集合,$p_{ij}=\frac{\exp\left(- \lVert Ax_i - Ax_j \rVert_2^2 \right)}{\sum_{k\neq i} \exp\left(- \lVert Ax_i - Ax_k \rVert_2^2 \right)}, j\neq i, p_{ii}=0$表示将第$i$个数据和其余所有样本的近邻概率分布(NN分类过程),距离越近其对应的$p_{ij}$越大,$f$的目标则是最大化留一验证近邻分类的准确性。$A\in \mathcal{R}^{d^\prime \times d}$是待优化的映射矩阵。试推导其梯度$\frac{\partial f}{\partial A}$。
	
	\part [10] 在自然语言处理领域,潜在语义分析(Latent Semantic Analysis, LSA)可以从文档-词矩阵中学习到文档表示、词表示,本质上也是对矩阵进行分解,试查阅相关资料,描述其具体步骤。并简述其与PCA的区别。
	
	\part [10] 根据局部线性嵌入(Locally Linear Embedding, LLE)的算法流程,尝试编写LLE代码,可以基于sklearn实现,并在简单数据集(``S"型构造数据或Mnist等)上进行实验,展示实验结果。
\end{parts}


\begin{solution}
	\begin{parts}
		\part 方便起见记$g_{ij}=exp(-||Ax_i-Ax_j||_2^2)$,此时$p_{ij}=\frac{g_{ij}}{\sum_{k\not=i}g_{ik}}$\\
		根据链式法则,一步一步求导$$\frac{\partial p_{ij}}{\partial A}=\frac{1}{(\sum_{k\not=i}g_{ik})^2}(\frac{\partial g_{ij}}{\partial A}\sum_{k\not=i}g_{ik}-g_{ij}\sum_{k\not=i}\frac{\partial g_{ik}}{\partial A})\ \ \ (1)$$
		再求$g_{ij}$对$A$的偏导$$\frac{\partial g_{ij}}{\partial A}=-2g_{ij}A(x_i-x_j)(x_i-x_j)^T\ \ \ \ \ \ \ \ \ \ \ \ \ \ \ \ \ \ \ \ (2)$$
		然后再求$f$对$A$的偏导$$\frac{\partial f}{\partial A}=\sum\limits_{i=1}^n\sum\limits_{j\in C_i}\frac{\partial p_{ij}}{\partial A}$$
		将$(2)$式代入$(1)$消去$g_{ij}$后再把$p_{ij}$的偏导代入梯度表达式,即可得到要求的梯度。\\
		
		此处省略化简过程,直接给出计算的结果
		$$\frac{\partial p_{ij}}{\partial A}=-2p_{ij}A[(x_i-x_j)(x_i-x_j)^T-\sum_{k\not=i}p_{ik}(x_i-x_k)(x_i-x_k)^T]$$
		$$
		\frac{\partial f}{\partial A}=2A\sum\limits_i[p_i\sum_{k\not=i}p_{ik}(x_i-x_k)(x_i-x_k)^T-\sum\limits_{j\in C_i}p_{ij}(x_i-x_j)(x_i-x_j)^T]
		$$
		\part 首先,要介绍话题向量空间,对于单词-文本矩阵$X$,他构成原始的
		单词向量空间,每一列是一个文本在单词向量空间中的表示。假设所有文本共包含了k个话题,每个话题由一个定义在单词集合$W$上的m维向量表示(即话题向量),那么
		这k个话题向量就张成了一个话题向量空间。\\

		潜在语义分析算法,就是对单词-文本矩阵进行奇异值分解,将其左矩阵作为话题向量空间,
		对角矩阵和右矩阵的乘积作为文本在话题向量空间中的表示。其步骤大致如下:\\
		(1)、给定文本集合$D=\{d_1,...,d_n\}$和单词集合$W=\{w_1,...w_m\}$,将其表示成
		一个单词-文本矩阵:$$X=
\left[
\begin{matrix}
	x_{11} & x_{12} & ... & x_{1n}\\
	x_{21} & x_{22} & ... & x_{2n}\\
	...    & ...    & ... & ...\\
	x_{m1} & x_{m2} & ... & x_{mn}
\end{matrix}
\right]
		$$
		其中,$x_{ij}$表示单词$w_i$在文本$d_j$中出现的权值\\
		(2)、根据给定的话题数k对单词-文本矩阵$X$进行截断奇异值分解$$
		X\approx U_k\Sigma_kV_k^T=[u_1 ... u_k ]
\left[
	\begin{matrix}
		\sigma_1 & 0 & 0 & ... \\
		0 & \sigma_2 & 0 & ... \\
		0 & 0 & \ddots & 0\\
		0 & 0 & ... & \sigma_k
	\end{matrix}		
\right]
\left[
	\begin{matrix}
		v_1^T\\
		v_2^T\\
		\vdots\\
		v_k^T
	\end{matrix}		
\right]
$$
在该式子中,$k\leq n\leq m$,$U_K$是$m*k$矩阵,它的列由$X$的前k个互相正交的左奇异向量
组成,$\Sigma_k$是k阶对角方阵,对角元素为前k个最大奇异值,$V_k$是$n*k$矩阵,它的列由$X$的
前k个互相正交的右奇异向量组成。\\

$U_k$中的每一个列向量表示一个话题(即话题向量),这k个话题向量张成一个子空间$U_k$就称为
话题向量空间。\\

接下来考虑文本在话题空间的表示,由截断奇异值分解:$X\approx U_k\Sigma_kV_k^T$,
故$X$的第j列向量$x_j$满足$$x_j\approx U_k(\Sigma_kV_k^T)_j=\sum\limits_{l=j}^k\sigma_lv_{jl}u_l,\ j=1,...,n$$
其中$(\Sigma_kV_k^T)_j$是矩阵$(\Sigma_kV_k^T)$的第j列向量,上式是文本$d_j$的近似表达式,
由k个话题向量的线性组合表示。对于矩阵$(\Sigma_kV_k^T)$,其每一个列向量都是一个文本在话题向量空间的表示\\

LSA与PCA的区别:\\
(1)、LSA本质上做的就是奇异值分解,只是对分解后的各个矩阵赋予了实际意义的解释;而PCA中,
奇异值分解只是一种求解方法,并没有涉及到其本质。\\

(2)、LSA是寻找F范数中的最佳线性子空间,PCA则是寻找最佳仿射线性子空间。
	\end{parts}
\end{solution}

\question [15] \textbf{特征选择基础}

Relief算法中,已知二分类问题的相关统计量计算公式如下:
\begin{equation} \delta^{j}=\sum_{i}-\operatorname{diff}\left(x_{i}^{j}, x_{i, n h}^{j}\right)^{2}+\operatorname{diff}\left(x_{i}^{j}, x_{i, \mathrm{nm}}^{j}\right)^{2}
\end{equation}
多分类的Relief-F算法的相关统计量计算公式如下:
\begin{equation}
	\delta^{j}=\sum_{i}-\operatorname{diff}\left(x_{i}^{j}, x_{i, \mathrm{nh}}^{j}\right)^{2}+\sum_{l \neq k}\left(p_{l} \times \operatorname{diff}\left(x_{i}^{j}, x_{i, l, \mathrm{nm}}^{j}\right)^{2}\right)
\end{equation}
其中$p_l$为第$l$类样本在数据集$D$中所占的比例。然而仔细观察可发现,二分类问题中计算公式的后一项$\operatorname{diff}\left(x_{i}^{j}, x_{i, \mathrm{nm}}^{j}\right)^{2}$的系数为1,多分类问题中后一项系数求和小于1,即$\sum_{l \neq k}p_l = 1-p_k < 1$。基于这个发现,请给出一种Relief-F算法的修正方案。


\begin{solution}
	将$p_l$替换为$\frac{p_l}{1-p_k}$,此时多分类问题后一项的系数为$\sum\limits_{l\not=k}\frac{p_l}{1-p_k}=\frac{\sum\limits_{l\not=k}p_l}{1-p_k}=\frac{1-p_k}{1-p_k}=1$,修正完成。
\end{solution}

\question [15] \textbf{特征选择拓展}

本题借助强化学习背景,主要探讨嵌入式选择在强化学习中的应用。强化学习可以看作一种最大化奖励(也就是目标)的机器学习方法,目的是学习到一个策略,使得执行这个策略获得的奖励值最大。基于TRPO(一种强化学习方法)的近似方法的近似问题如下
\begin{equation}
	\begin{array}{cc}
		\max\limits_{\theta} & \left(\nabla L_{\theta_{\text {old }}}(\theta)\right)^{T}\left(\theta-\theta_{\text {old }}\right) \\
		\text { s.t. } & \frac{1}{2}\left(\theta-\theta_{\text {old }}\right)^{T} H\left(\theta-\theta_{\text {old }}\right) \leq \delta
	\end{array}
\end{equation}
这里采用了参数化表示方法,其中$\theta$表示新策略,$\theta_{\text {old }}$表示旧策略,方法需要通过策略的目标函数$L_{\theta_{\text {old }}}$来更新旧策略,最终目标是学习到最大化目标函数的新策略。这里要最大化的表达式可以对应理解为最小化损失函数,即类似于课本252页式(11.5)。\\
如果将目标$L$分解为很多个子目标,即$L=\left[L_{1}, L_{2}, \ldots, L_{n}\right]^{T}$,每个目标对应相应的权重$w=\left[w_{1}, w_{2}, \ldots, w_{n}\right]^{T}$,新方法(称为ASR方法)的优化目标如下
\begin{equation}
	\begin{array}{cl}
		\max _{w} & \max _{\theta}\left(\nabla\left(L^{T} w\right)\right)^{T}\left(\theta-\theta_{\mathrm{old}}\right) \\
		\text { s.t. } & \frac{1}{2}\left(\theta-\theta_{\mathrm{old}}\right)^{T} H\left(\theta-\theta_{\mathrm{old}}\right) \leq \delta \\
		& \|w\|_{1}=1 \\
		& w_{i} \geq 0, \quad i=1,2, \ldots, n
	\end{array}
\end{equation}

问:
\begin{parts}
	\part [10] 尝试分析ASR方法中加入$w$的L1范数约束的现实意义。(提示:不同目标对应的参数$w_i$是需要学习的参数。原目标$L$现由多个子目标组成,每个子目标的质量良莠不齐)
	\part [5] 在ASR方法基础上提出的BiPaRS方法解除了$w$的L1范数这一限制,使得更多样$w$可以出现、更多种$L$可以被使用。结合这一点,论述特征选择需要注意的事项。
\end{parts}

\begin{solution}
\begin{parts}
	\part 加入$w$后,我们对每个$L_i$就赋予了一个权重,通过调整不同质量的子目标的权重,可以使得优化更具有针对性,效果也更好。\\
	使用$w$的$L_1$范数进行约束,计算上更加简便,同时也更便于理解$w$的意义。
	\part 在特征选择中,我们需要注意保留特征的数目,即不能过多,又不能过少。\\
	
	选取的特征过多时,可能仍然存在一部分冗余特征,不仅不能有效降低训练模型的开销,
	对模型的准确度也有负面影响;此外,由于去除的特征数量少,也无法完全规避过拟合的风险。\\

	选取的特征过少时,很容易出现欠拟合问题,同样不利于提升模型的准确度。
\end{parts}
	
\end{solution}


\end{questions}

\end{document}